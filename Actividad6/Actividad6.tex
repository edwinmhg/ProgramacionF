\documentclass{article}

% set font encoding for PDFLaTeX or XeLaTeX
\usepackage{graphicx}
\usepackage{ifxetex}
\ifxetex
  \usepackage{fontspec}
\else
  \usepackage[T1]{fontenc}
  \usepackage[utf8]{inputenc}
  \usepackage{lmodern}
  
\fi

% used in maketitle
\title{Actividad 6}
\author{Edwin Herrera\\
Departamento de Fisica \\
Universidad de Sonora}
\date{30 de Noviembre de 2017}
% Enable SageTeX to run SageMath code right inside this LaTeX file.
% documentation: http://mirrors.ctan.org/macros/latex/contrib/sagetex/sagetexpackage.pdf
% \usepackage{sagetex}

\begin{document}
\maketitle
\clearpage
\section{Sistema Sol-Tierra}
La traslación de la Tierra es el movimiento de este planeta alrededor del Sol, que es la estrella central del sistema solar. La Tierra describe a su alrededor como una órbita elíptica.

Si se toma como referencia la específica posición de una estrella, la Tierra realiza una vuelta completa en un año sidéreo, cuya duración es de 365 días, 5 horas, 45 minutos y 46 segundos. El año sidéreo es de poca importancia práctica. Para las actividades terrestres es más importante la medición del tiempo según las estaciones.

\subsection{Sistema Tierra-Luna}
La Luna es el único satélite natural de la Tierra. Con un diámetro ecuatorial de 3474 km es el quinto satélite más grande del Sistema Solar, mientras que en cuanto al tamaño proporcional respecto de su planeta es el satélite más grande: un cuarto del diámetro de la Tierra y 1/81 de su masa. 

Al desplazarse en torno del Sol, la Tierra arrastra a su satélite y la forma de la trayectoria que esta describe es una curva de tal naturaleza que dirige siempre su concavidad hacia el Sol. La velocidad con que la Luna se desplaza en su órbita alrededor de la Tierra es de 1 km/s.

\subsection{Trabajo Sistema Sol-Tierra-Luna}
En este trabajo tuvimos que hacer un modelo Sol-Tierra-Luna en el cual la orbita de estas eran una circunferencia.Teniamos que hacer un programa en el cual la orbita de la tierra alrededordel sol y a su ves la orbita de la luna alrededor de la tierra. Teniamos que encontrar la posicion en formas polares para dar el resultado en coordenadas cartesianas. 

\clearpage
\begin{verbatim}
El código que utilice para programar en Fortran 90 fue el siguiente:

function solx(thetaL) result (x)
  double precision, intent(in) :: thetaL
  double precision :: x
  double precision, parameter :: Rsol = 1.49d8
  x = Rsol * dcos(thetaL)
end function solx

function soly(thetaL) result (y)
  double precision, intent(in) :: thetaL
  double precision :: y
  double precision,parameter :: Rsol = 1.49d8
  y = Rsol * dsin(thetaL)
end function soly

subroutine luna(Rsol, Rluna, Px, Py, thetaL, thetaS)
   double precision, intent (in) :: Rsol, thetaL, thetaS
   double precision, intent (out) :: Px, Py
   double precision :: Rluna
   Rluna = Rsol / 4.0d0
   Px = (Rsol * dcos(thetaS)) + (Rluna * dcos(thetaL))
   Py = (Rsol * dsin(thetaS)) + (Rluna * dsin(thetaL))
 
end subroutine luna 

program begin

  implicit none
  double precision :: fi, fj, Rsol, Rluna, Px, Py, thetaL, rd, dia
  double precision :: Vluna, Vsol, solx, soly, thetaS
  double precision, parameter :: pi=3.1416d0, Tsol = 360, Tluna = 28 
  integer :: j
  double precision, dimension(360) :: xtotal, ytotal
  double precision, dimension(360) :: x, y

  !Convertimos los angulos a radianes.
  rd = pi / 180d0
  !Definimos el radio del sol.
  Rsol = 1.496d8
  !Los días que pasan por 1 radian.
  dia = 365.26d0 / (360d0*rd)
  !Velocidad de la Luna.
  Vluna = 2d0 * (pi / Tluna)
  !Velocidad del Sol.
  Vsol = 2d0 * (pi / Tsol)

  
   open (unit=1, file = 'Luna-Tierra.dat', status = 'unknown')
   open (unit=2, file = 'Sol-Tierra.dat', status = 'unknown')

do j=1, 360, 1
   fj = dble(j)
   thetaS = fj * Vsol
   thetaL = fj * Vluna
   
   x(j)= solx(thetaS)
   y(j)= soly(thetaS)
   

   call luna(Rsol, Rluna, Px, Py, thetaL, thetaS)
   xtotal(j) = Px
   ytotal(j) = Py

     write (1,*) xtotal(j), ytotal(j)
     write (1,*) ' '
     write (2,*) x(j), y(j)
     write (2,*) ' '
     
end do
     
  close (unit=1)
  close (unit=2)
  
end program
\end{verbatim}
\begin{figure}[h!]
  \includegraphics[width=\linewidth]{Sol-Tierra-Luna.png}
  \caption{Grafica de posiciones de las orbitas}
  \label{fig:Grafica}
\end{figure}
\end{document}
