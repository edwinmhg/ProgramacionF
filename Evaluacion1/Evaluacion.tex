\documentclass{article}

% set font encoding for PDFLaTeX or XeLaTeX
\usepackage{ifxetex}
\ifxetex
  \usepackage{fontspec}
\else
  \usepackage[T1]{fontenc}
  \usepackage[utf8]{inputenc}
  \usepackage{lmodern}
\fi

% used in maketitle
\title{Evaluación}
\author{Edwin Herrera \\
Departamento de Fisica \\
Universidad de Sonora}
\date{30 de Octubre de 2017}

% Enable SageTeX to run SageMath code right inside this LaTeX file.
% documentation: http://mirrors.ctan.org/macros/latex/contrib/sagetex/sagetexpackage.pdf
% \usepackage{sagetex}

\begin{document}
\maketitle
\section{Introducción}
El examen consistió en tres problemas, en cada uno de ellos nos proporcionaban un problema ejemplo, el cual teniamos que analizar y ver como funcionaba y de ahi tener una base para desarrollar el verdadero problema.

\subsection{Problema 1}

En el problema 1 nos daban un codigo de ejemplo que srive para calcular la superficie de un cilindro.
Se tenia que modificar para calcular el area y la superficie de una esfera. El codigo resultante fue esta:
\begin{verbatim}
program sphere

! Calculate the surface area of a cylinder.
!
! Declare variables and constants.
! constants=pi
! variables=radius squared and height

  implicit none

! Require all variables to be explicitly declared

  integer :: ierr
  character(1) :: yn
  real :: radius, area, volume
  real, parameter :: pi = 3.141592653589793

  interactive_loop: do

!   Prompt the user for radius
!   and read them.

    write (*,*) 'Enter radius.'
    read (*,*,iostat=ierr) radius

!   If radius  could not be read from input,
!   then cycle through the loop.

    if (ierr /= 0) then
      write(*,*) 'Error, invalid input.'
      cycle interactive_loop
    end if

!   Compute area.  The ** means "raise to a power."

    area = 4 * pi * (radius ** 2) 
    volume = (4 * pi * (radius ** 3) / 3)
    

!   Write the input variable (radius)
!   and output (area and volume) to the screen.

    write (*,'(1x,a7,f6.2,5x,a7,f8.2,5x,a5,f8.2)') &
      'radius=',radius,'area=',area, 'volume=',volume

    yn = ' '
    yn_loop: do
      write(*,*) 'Perform another calculation? y[n]'
      read(*,'(a1)') yn
      if (yn=='y' .or. yn=='Y') exit yn_loop
      if (yn=='n' .or. yn=='N' .or. yn==' ') exit interactive_loop
    end do yn_loop

  end do interactive_loop

end program sphere
\end{verbatim}

\subsection{Problema 2}
En el segundo problema nos proporcionaban un codigo para calcular una sumatoria, igual lo tenemos que analizar y entender el codigo para de ahi hacer otro que pueda calucar la media aritmetica y media armonica.
Este es el codigo resultante:
\begin{verbatim}
program summation
implicit none
integer :: j
real :: mean, a, i, mean2

print*, "This program performs Arithmetic and Harmonic means. Enter 0 to stop."
open(unit=10, file="MeanData.DAT")
mean2 = 0
mean = 0
i=0
  
do
 print*, "Add:"
 read*, a
 if (a == 0) then
  exit
   else
   
mean = ( mean + a )   
i = i + 1

mean2 = mean2 + (1/a)

end if

end do
mean = mean / i
mean2 = 1 / (mean2 /i)
 write(10,*) a, i

print*, "Media aritmetica =", mean, "i =", i,"Media armonica =", mean2
write(10,*) "Media aritmetica =", mean, "Media armonica =", mean2
close(10)

end program
\end{verbatim}
\begin{verbatim}
El programa asi funciona:  This program performs Arithmetic and Harmonic means.
Enter 0 to stop.
 Add:
1
 Add:
2
 Add:
3
 Add:
4
 Add:
5

 Add:
0
 Media aritmetica =   3.00000000     i =   5.00000000     
 Media armonica =   2.18978071    
\end{verbatim}
\end{document}
