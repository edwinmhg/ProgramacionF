\documentclass{article}

% set font encoding for PDFLaTeX or XeLaTeX
\usepackage{ifxetex}
\ifxetex
  \usepackage{fontspec}
\else
  \usepackage[T1]{fontenc}
  \usepackage[utf8]{inputenc}
  \usepackage{lmodern}
\fi

% used in maketitle
\title{Evaluación 2}
\author{Edwin Herrera\\
Universidad de Sonora\\
Departamento de Física}

% Enable SageTeX to run SageMath code right inside this LaTeX file.
% documentation: http://mirrors.ctan.org/macros/latex/contrib/sagetex/sagetexpackage.pdf
% \usepackage{sagetex}

\begin{document}

\maketitle
\clearpage
\section{Primera actividad}
1.- Se proporciona el siguiente código, que utiliza una función en Fortran 90 para la Serie de Maclaurin function exptaylor(x,n)para aproximar la función exponencial f(x) = exp(x), en el punto x=1, utilizando n=20 términos de la serie.

\begin{verbatim}
! ----------- Begin ------------

!taylor.f90

program taylor

    implicit none                  
real (kind=8) :: x, exp_true, y
    real (kind=8), external :: exptaylor
    integer :: n

    n = 20               ! number of terms to use
    x = 1.0
    exp_true = exp(x)
    y = exptaylor(x,n)   ! uses function below
    print *, "x = ",x
    print *, "exp_true  = ",exp_true
    print *, "exptaylor = ",y
    print *, "error     = ",y - exp_true

end program taylor

!==========================
function exptaylor(x,n)
!==========================
    implicit none

    ! function arguments:
    real (kind=8), intent(in) :: x
    integer, intent(in) :: n
    real (kind=8) :: exptaylor

    ! local variables:
    real (kind=8) :: term, partial_sum
    integer :: j

    term = 1.
    partial_sum = term

    do j=1,n
        ! j'th term is  x**j / j!  which is the previous term times x/j:
        term = term*x/j   
        ! add this term to the partial sum:
        partial_sum = partial_sum + term   
        enddo
     exptaylor = partial_sum  ! this is the value returned
end function exptaylor

! --------  End -------------

\end{verbatim}

\subsection{Explicación}
En el programa se utilizó una serie de Maclaurin que es la de aproximar el exponencial.
Los resultados que dió al correr el programa fueron los siguientes:
\begin{verbatim}
 x =    1.0000000000000000     
 exp_true  =    2.7182818284590451     
 exptaylor =    2.7182818284590455     
 error     =    4.4408920985006262E-016

Al analizarlo nos damos cuenta que la función se aproximo en el punto x=1 y n=20 
terminos en la serie.

exp_true Es un valor de la funcion exponencial y exptaylor es el valor de la
funcion que calculamos en n=20, y el error es la resta de exp_true menos 
exptaylor. Es el error que nos da porque no se aproximo al infinito.
\end{verbatim}



\end{document}
