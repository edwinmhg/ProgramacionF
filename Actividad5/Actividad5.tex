\documentclass{article}

% set font encoding for PDFLaTeX or XeLaTeX
\usepackage{graphicx}
\usepackage{ifxetex}
\ifxetex
  \usepackage{fontspec}
\else
  \usepackage[T1]{fontenc}
  \usepackage[utf8]{inputenc}
  \usepackage{lmodern}
  
\fi

% used in maketitle
\title{Actividad 5}
\author{Edwin Herrera\\
Departamento de Fisica \\
Universidad de Sonora}
\date{13 de Noviembre de 2017}

% Enable SageTeX to run SageMath code right inside this LaTeX file.
% documentation: http://mirrors.ctan.org/macros/latex/contrib/sagetex/sagetexpackage.pdf
% \usepackage{sagetex}

\begin{document}
\maketitle
\clearpage
\section{Sistema Sol-Tierra}
La traslación de la Tierra es el movimiento de este planeta alrededor del Sol, que es la estrella central del sistema solar. La Tierra describe a su alrededor como una órbita elíptica.

Si se toma como referencia la específica posición de una estrella, la Tierra realiza una vuelta completa en un año sidéreo, cuya duración es de 365 días, 5 horas, 45 minutos y 46 segundos. El año sidéreo es de poca importancia práctica. Para las actividades terrestres es más importante la medición del tiempo según las estaciones.

\subsection{Trabajo}
En este trabajo tuvimos que hacer un modelo Sol-Tierra en el cual la orbita era una circunferencia. Teniamos que encontrar la posicion en formas polares para dar el resultado en coordenadas cartesianas.

\begin{figure}[h!]
  \includegraphics[width=\linewidth]{Circunferencia.png}
  \caption{Grafica de posiciones de la tierraa}
  \label{fig:Grafica}
\end{figure}

EL codigo que utilicé para programar en Fortran 90 fue:
\begin{verbatim}
program begin

  implicit none
  double precision :: fi
  double precision, parameter :: r=1.496d8, pi=3.1416d0  
  integer :: i
  integer, parameter :: ntimes = 360
  double precision, dimension(1000) :: x, y
  
  
 
  open (unit=1, file ='datos.dat', status = 'unknown')
  
  do i = 1, ntimes, 1
     fi = dble(i)

!Convertimos el angulo a radianes.
     fi = fi * pi / 180d0
    
!Calculamos la posicion.

     x(i) = r * dcos(fi)      
     y(i) = r * dsin(fi) 

     write (1,*) x(i), y(i)
     write (1,*) ' '
     
  end do
  
  close (unit=1)
  
  end program
\end{verbatim}

\end{document}
