\documentclass{article}

% set font encoding for PDFLaTeX or XeLaTeX
\usepackage{ifxetex}
\ifxetex
  \usepackage{fontspec}
\else
  \usepackage[T1]{fontenc}
  \usepackage[utf8]{inputenc}
  \usepackage{lmodern}
\fi

% used in maketitle
\title{Actividad 2}
\author{Edwin Herrera \\
Departamento de Fisica \\
Universidad de Sonora}


% Enable SageTeX to run SageMath code right inside this LaTeX file.
% documentation: http://mirrors.ctan.org/macros/latex/contrib/sagetex/sagetexpackage.pdf
% \usepackage{sagetex}

\begin{document}
\maketitle
\clearpage

\section{Movimiento de un proyectil}

El movimiento de un proyectil es una forma de movimiento en la cual el objeto o particula se tira cerca de la superficie de la tierra, y se mueve a lo largo de una curva en interaccion con la gravedad. La unica fuerza significativa que actua sobre el objeto es la gravedad, que "jala hacia abajo" aplicando una aceleración. En consecuencia de la inercia del objeto, no se necesitan fuerzas externas horizontales para mantener la velocidad horizontal del objeto.

\subsection{Prueba y error}

En este paso comprobamos que el angulo de 45 grados es el que daba un alcance maximo. Como se puede ver en la tabla 1.
Y se calculó con la formula:
$$x = v_{0} t \cos \theta$$

\begin{center}
\begin{table}[]
\centering
\caption{Tiro de Proyectil}
\label{my-label}
\begin{tabular}{|l|l|l|l|l|l|l|}
\hline
Angulo & Tiempo & Velocidad Inicial & x    & y     & vf   & Theta  \\ \hline
35     & .2     & .1                & 1.6  & -0.18 & 1.9  & -87.53 \\ \hline
45     & .2     & .1                & 1.4  & -0.18 & 1.89 & -87.85 \\ \hline
55     & .2     & .1                & 1.14 & -0.17 & 1.87 & -88.25 \\ \hline
\end{tabular}
\end{table}
\end{center}

\subsection{Tiempo de vuelo}

EL tiempo total t en el que el proyectil permanece en el aire se le llama tiempo total de vuelo. Y se calcula con la formula:
$$t = \frac{2 v_{0} \sin \theta}{g}$$
Y los datos estan registrados en la tabla 2.

\begin{table}[]
\centering
\caption{Tiempo de vuelo}
\label{my-label}
\begin{tabular}{|l|l|l|}
\hline
Rapidez Inicial & Ángulo & Tiempo \\ \hline
15              & 60     & 2.65   \\ \hline
5               & 130    & .781   \\ \hline
120             & 20     & 8.37   \\ \hline
100             & 100    & 20.09  \\ \hline
\end{tabular}
\end{table}

\subsection{Altura maxima}
La altura maxima que el objeto alcanzará es conocida como el el pico de la altura maxima.
Se utiliza la formula:
$$h = \frac {v_0^2 \sin^2 \theta}{2g} $$
Y los datos estan registrados en la tabla 3.

\begin{table}[]
\centering
\caption{Altura maxima}
\label{my-label}
\begin{tabular}{|l|l|l|}
\hline
Rapidez Inicial & Ángulo & Altura Maxima \\ \hline
15              & 60     & 8.60          \\ \hline
5               & 130    & .748          \\ \hline
120             & 20     & 85.94         \\ \hline
100             & 100    & 494.81        \\ \hline
\end{tabular}
\end{table}

\subsection{Desplazamiento maximo}

La distancia maximca recorrida cuando el objeto termina o en cualquier parte de  su trayectoria se calcula con la formula:
$$x = v_{0} t \cos \theta$$
Y los datos estan registrados en la tabla 4.

\begin{table}[]
\centering
\caption{Distancia maxima}
\label{my-label}
\begin{tabular}{|l|l|l|}
\hline
Rapidez Inicial & Ángulo & Distancia maxima \\ \hline
15              & 60     & 19.88            \\ \hline
5               & 130    & -2.51            \\ \hline
120             & 20     & 944.5            \\ \hline
100             & 100    & -343.00          \\ \hline
\end{tabular}
\end{table}


\end{document}
