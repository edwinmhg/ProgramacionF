\documentclass{article}

% set font encoding for PDFLaTeX or XeLaTeX
\usepackage{ifxetex}
\ifxetex
  \usepackage{fontspec}
\else
  \usepackage[T1]{fontenc}
  \usepackage[utf8]{inputenc}
  \usepackage{lmodern}
\fi

% used in maketitle
\title{Actividad 1}
\author{Edwin Herrera \\
Departamento de Fisica \\
Universidad de Sonora}
\date{31 de Agosto de 2017}

% Enable SageTeX to run SageMath code right inside this LaTeX file.
% documentation: http://mirrors.ctan.org/macros/latex/contrib/sagetex/sagetexpackage.pdf
% \usepackage{sagetex}

\begin{document}
\maketitle
% Son comentarios
\section{Introducción}
Se pide elaborar una bitácora de los comandos que vas a ir probando en los puntos del 1 al 8 del Tutorial de Linux, con la intención de elaborar un pequeño manual con los primeros comandos de Linux para principiantes, diseñado y sintetizado por ti y para ti. La estructura será: comando, descripción breve y un ejemplo de entrada y salida.

\subsection{Comandos de bash}
% Pegamos lo copiado que escribimos en Emacs


\subsubsection{Command Line}
\begin{itemize}
\item edwin@bash \\
Es la terminal en la que estamos ubicados.

\end{itemize}
\subsubsection{Basic Navigation}
\begin{itemize}
\item pwd \\
Te dice en que directorio estas trabajando.

\item ls \\
Te dice el contenido del directorio en el que estas trabajando.

\item ls -l \\
Te da un listado de todos los archivos que estan en la carpeta o directorio que estas ubicado en lista larga.
\end{itemize}
\subsubsection{More About Files}
\begin{itemize}
\item file \\
Es para saber que tipo de documentos tienes en ese directorio.

\item ls -a \\
Es para enlistar el contenido del directorio incluyendo los archivos escondidos.
\end{itemize}
\subsubsection{Manual Pages}
\begin{itemize}
\item man <search term> \\
Ver el manual de como usar un comando.

\item man -k <search term> \\
Te da un listado de todos los manuales que contienen el termino buscado.
\end{itemize}
\subsubsection{File Manipulation}
\begin{itemize}

\item mkdir \\
Crean un directorio nuevo.

\item rmdir \\
Borrar directorio.

\item touch \\
Crear un archivo en blanco.

\item cp \\
Copiar un archivo o un directorio.

\item mv \\
Es para mover un archivo o un directorio.

\item rm \\
Es para borrar un archivo.
\end{itemize}
\subsubsection{Vi Text Editor}
\begin{itemize}
\item vi \\
Es para editar un archivo.

\item cat \\
Es para ver un archivo.
\end{itemize}
\subsubsection{Wildcards}
\begin{itemize}

\item * \\
Representa cero o mas caracteres.

\item ? \\
Respresenta un solo caracter.

\item $[ $ $ ]$ \\
Representa un rango de caracteres.

\item $\hat{ }$ \\
Es para especificar cuales caracteres no quieres que esten en el resultado.
\end{itemize}
\subsubsection{Permissions}
\begin{itemize}

\item chmod \\
Cambiar los permisos de un archivo o un directorio.

\item ls -ld \\
Ver los permisos 

\end{itemize}
\end{document}



